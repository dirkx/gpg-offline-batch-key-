\documentclass[11pt,svgnames]{report}
\usepackage{geometry}                % See geometry.pdf to learn the layout options. There are lots.
\geometry{a4paper}                   % ... or a4paper or a5paper or ... 
\usepackage{graphicx}
\usepackage{listings}
\usepackage{epstopdf}
\usepackage{watermark}
\usepackage{lipsum}
\usepackage{tikz}
\usepackage{kpfonts}
\usepackage{draftwatermark}
\usepackage[bottom]{footmisc}
\usepackage{fancyhdr}
\pagestyle{fancy}

\SetWatermarkText{$SPECIMEN}

\newcommand{\topbar}[2]{
 {\begin{tikzpicture}[remember picture,overlay]
    \node[yshift=-2cm] at (current page.north west)
      {\begin{tikzpicture}[remember picture, overlay]
        \draw[fill=#1] (0,0) rectangle
          (\paperwidth,2cm);
        \node[anchor=east,xshift=.9\paperwidth,rectangle,
              rounded corners=20pt,inner sep=11pt,
              fill=MidnightBlue]
              {\color{white}\normalfont\sffamily\Huge\bfseries\scshape #2};
       \end{tikzpicture}
    };
   \end{tikzpicture}
  }
}


\renewcommand{\headrulewidth}{0pt}
\newcommand{\pagefooter}{\scriptsize \textcolor{gray}{p\thepage\ of 10 - \date - 1.03
\ifdefined\projectCode%
- \textsc{\projectCode}%
\fi%
}%
}%
\cfoot{}
\rfoot{\pagefooter}
\lfoot{\scriptsize \textcolor{gray}{Web Weaving, Janvossensteeg 37-39, 2312WC Leiden / office@webweaving.org }}

\fancypagestyle{plain}{
    \fancyhf{}
    \fancyfoot[R]{\pagefooter}
        \renewcommand{\headrulewidth}{0pt}
}

\begin{document}
\watermark{\topbar{green}{public}}

\chapter*{Recovery document for key {$KEYNAME}}
\section*{$DATE}

This document contains the recovery procedure for the \emph{private} key whose public key has the fingerprint shown below:
'
\lstinputlisting[breaklines,tabsize=2,basicstyle=\ttfamily\small]{gpg.txt}

\section*{Process}

Prior to opening the sealed envelope with the private key details we suggest that you:

\begin{enumerate}
\item Consider having a second person acting as an observer; and document the steps taken during recovery.
\item Ensure that you have a clean machine; and that your pgp, gpg or gnupg supports ``ECDSA'' as a pubkey (Use ``gpg --version'' to check).
\item Have acces to a QR decoder; or have a library such as https://github.com/dlbeer/quirc.git compiled and tested. (The alternative is OCR or entering half a page (500 numbers) by hand).
\end{enumerate}

\pagebreak 
\section*{Decode}
\label{quirc}

Unfortunately most QR decoders are somewhat careless about binary data and loose the `top bit' Code known to correctly extract QR binary code is shown below. Needs to be linked against: 

\begin{quotation}
https://github.com/dlbeer/quirc.git (-- v1.00, commit 307473d on 2018-02-01).
\end{quotation}

\lstinputlisting[breaklines,language=sh,tabsize=2,basicstyle=\ttfamily\tiny]{$CODE}
\pagebreak
\subsection*{Compile and extract raw key}

A typical sequence of steps to compile the decoder and import thee key into a GPG
is shown below. The 'XXXX' refers to the path to the C code listed on the previous page. The `generation' script on the next page does much the same thing in the section \emph{`Decode the key and verify ...'}.

\lstinputlisting[breaklines,language=sh,tabsize=2,basicstyle=\ttfamily\tiny]{$EXTRACT}

\pagebreak 

\section*{Script used to generate this key}
\texttt{ \lstinputlisting[breaklines,language=sh,tabsize=2,basicstyle=\ttfamily\tiny]{$SCRIPT} }
\pagebreak 

\section*{GPG listing}

\lstinputlisting[breaklines,tabsize=2,basicstyle=\ttfamily\tiny]{gpg.txt}

\section*{Auxilary info} \lstinputlisting[breaklines,tabsize=2,basicstyle=\ttfamily\tiny]{info.txt}

\chapter*{Private key ${KEYNAME}}
\section*{$DATE}
This document allows for the reconstruction of a private key (secret key).

The path is as follows:

\begin{enumerate}
\item Ensure that you have a clean machine; and that your pgp, gpg or gnupg supports ``ECDSA'' as a pubkey (Use ``gpg --version'' to check).
\item Scan in the private key on page \pageref{qr}. 
\item Using a QR decoder; extract the binary data from it\footnote{Note that a fair number of qrdecoders on the internet are `broken' due to careless handling of UTF8 and binary data. See page \pageref{quirc} for a tested alternative.}.

If that does not work - enter the keys manaully (provided on the next pages).
\item (optional) Check if the entered/extracted key is correct using the SHA256 listed on each page.
\item Import the extracted secret key with ``gpg2 --homedir . --import qr-code-extrracted.binary''.
\end{enumerate}

You can now use the keyring created in the directory specified with ``--homedir'' to decrypt files.

Note that this secret key is \emph{not} protected by a password.

\watermark{\topbar{red}{sensitive key material}}
\pagebreak 

\section*{Private key ${KEYNAME} - $DATE}
\label{qr}
\includegraphics[width=16cm]{qr.png}

\section*{Sha 256 of private key} \texttt{$SHA256}

\pagebreak
\section*{Private key ${KEYNAME} - $DATE} {\small \texttt{ \lstinputlisting[breaklines,language=sh,tabsize=2]{gpg.priv.raw.asc} }}

\section*{Sha 256 of the ASCII armoured  private key including headers}
\texttt{$SHA256A}
\section*{Sha 256 of private key}
\texttt{$SHA256}
\pagebreak 

\section*{Private key ${KEYNAME} - $DATE}
\texttt{ \lstinputlisting[breaklines,language=sh,tabsize=2]{priv.txt} }
\section*{Sha 256 of private key}
\texttt{$SHA256}
\end{document}

